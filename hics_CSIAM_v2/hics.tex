\documentclass[mathpazo]{csam}

\RequirePackage{fix-cm}
% Packages and macros go here
\usepackage{mathrsfs,amsmath,amssymb,bm}
\usepackage{lipsum}
\usepackage{amsfonts}
\usepackage{epstopdf}
\usepackage{makecell, rotating}
\usepackage{multirow}
\usepackage{algorithm} 
\usepackage{algorithmic}
\usepackage{multirow}
\usepackage{subfigure}
\usepackage{xcolor}

\newcommand{\creflastconjunction}{, and~}
\newcommand{\bbR}{\mathbb{R}}
\newcommand{\bbC}{\mathbb{C}}
\newcommand{\bbZ}{\mathbb{Z}}
\newcommand{\bbQ}{\mathbb{Q}}
\newcommand{\bbN}{\mathbb{N}}
\newcommand{\by}{\bm{y}}
\newcommand{\bz}{\bm{z}}
\newcommand{\bx}{\bm{x}}
\newcommand{\tbx}{\tilde{\bm{x}}}
\newcommand{\ba}{\bm{a}}
\newcommand{\bb}{\bm{b}}
\newcommand{\bn}{\bm{n}}
\newcommand{\br}{\bm{r}}
\newcommand{\bh}{\bm{h}}
\newcommand{\bH}{\bm{H}}
\newcommand{\bN}{\bm{N}}
\newcommand{\bk}{\bm{k}}
\newcommand{\bo}{\bm{0}}
\newcommand{\bhphi}{\bm{\hat{\phi}}}
\newcommand{\bhpsi}{\bm{\hat{\psi}}}
\newcommand{\bphi}{\bm{{\phi}}}
\newcommand{\bt}{\bm{t}}
\newcommand{\bs}{\bm{s}}
\newcommand{\bu}{\bm{u}}
\newcommand{\bM}{\bm{M}}
\newcommand{\bP}{\bm{P}}
\newcommand{\bQ}{\bm{Q}}
\newcommand{\bB}{\bm{B}}
\newcommand{\bA}{\bm{A}}
\newcommand{\bI}{\bm{I}}
\newcommand{\calS}{\mathcal{S}}
\newcommand{\calB}{\mathcal{B}}
\newcommand{\calP}{\mathcal{P}}
\newcommand{\calR}{\mathcal{R}}
\newcommand{\calM}{\mathcal{M}}
\newcommand{\calJ}{\mathcal{J}}
\newcommand{\calF}{\mathcal{F}}
\newcommand{\calI}{\mathcal{I}}
\newcommand{\calD}{\mathcal{D}}
\newcommand{\calA}{\mathcal{A}}
\newcommand{\hF}{\hat{F}}
\newcommand{\bq}{\bm{q}}
\newcommand{\hw}{\hat{w}}
\newcommand{\hphi}{\hat{\phi}}
\newcommand{\hPhi}{\hat{\Phi}}
\newcommand{\hpsi}{\hat{\psi}}
\newcommand{\hPsi}{\hat{\Psi}}
\newcommand{\hxi}{\hat{\xi}}
\newcommand{\hJ}{\hat{J}}
\newcommand{\vzero}{\bm{0}}
\newcommand{\tE}{\tilde{E}}
\newcommand{\tF}{\tilde{F}}
\newcommand{\tG}{\tilde{G}}
\newcommand{\tg}{\tilde{g}}
\newcommand{\bg}{\bar{g}}
\newcommand{\tJ}{\tilde{\mathcal{J}}}
\newcommand{\td}{\tilde{d}}
\newcommand{\la}{\langle}
\newcommand{\ra}{\rangle}
\newcommand{\dom}{\mathrm{dom}}
\DeclareMathOperator{\intdom}{\mathrm{int}\dom}
\DeclareMathOperator{\ridom}{\mathrm{ri}\dom}
\DeclareMathOperator{\affdom}{\mathrm{aff}\dom}
\newcommand{\recheck}[1]{{\color{red}{#1}}}




\newcommand\tbbint{{-\mkern -16mu\int}}
\newcommand\tbint{{\mathchar '26\mkern -14mu\int}}
\newcommand\dbbint{{-\mkern -19mu\int}}
\newcommand\dbint{{\mathchar '26\mkern -18mu\int}}
\newcommand\bint{
	{\mathchoice{\dbint}{\tbint}{\tbint}{\tbint}}
}
\newcommand\bbint{
	{\mathchoice{\dbbint}{\tbbint}{\tbbint}{\tbbint}}
}

\allowdisplaybreaks[4]
\DeclareMathOperator*{\argmin}{\mathrm{argmin}}
\DeclareMathOperator*{\argmax}{\mathrm{argmax}}

\newcommand{\Prox}{\mathrm{Prox}}
\newcommand{\BProx}{\mathrm{BProx}}
\newcommand{\GProx}{\mathrm{GProx}}
\newcommand{\dist}{\mathrm{dist}}
%\newcommand{\mod}{\mathrm{mod}}

%\renewcommand{\baselinestretch}{1.5}
\newtheorem{thm}{Theorem}
\newtheorem{assumption}[thm]{Assumption}
% \newtheorem{theorem}{Theorem}[section]
% \newtheorem{lemma}[theorem]{Lemma}

% \theoremstyle{definition}
% \newtheorem{definition}[theorem]{Definition}
% \newtheorem{example}[theorem]{Example}
\newtheorem{xca}[theorem]{Exercise}

\theoremstyle{remark}
% \newtheorem{remark}[theorem]{Remark}


\newcommand{\Note}[1]{{\color{blue}{#1}}} % add by BAO for notice.
\newcommand{\Notsure}[1]{{\color{red}{#1}}} % add by BAO for notice.


%%%%% author macros %%%%%%%%%
% place your own macros HERE
%%%%% end %%%%%%%%%

\begin{document}
	%%%%% title : short title may not be used but TITLE is required.
	% \title{TITLE}
	% \title[short title]{TITLE}
	\title[Hill-climbing method with a stick]{An efficient method of finding a
	neighbourhood of minimizers}

	%%%%% author(s) :
	
	% multiple authors:
	% Note the use of \affil and \affilnum to link names and addresses.
	% The author for correspondence is marked by \corrauth.
	% use \emails to provide email addresses of authors
	% e.g. below example has 3 authors, first author is also the corresponding
	%      author, author 1 and 3 having the same address.
	\author[Yunqing Huang and Kai Jiang]{
	Yunqing Huang\affil{1}\corrauth~and Kai Jiang\affil{1} }
	\address
	{ \affilnum{1}\ School of Mathematics and Computational Science, 
		Hunan Key Laboratory for Computation and Simulation in Science and Engineering,
		Xiangtan University, Xiangtan, Hunan, 411105, China.}
	\emails{{\tt huangyq@xtu.edu.cn} 
		}
	% \footnote and \thanks are not used in the heading section.
	% Another acknowlegments/support of grants, state in Acknowledgments section
	% \section*{Acknowledgments}
	
	%%%%% Begin Abstract %%%%%%%%%%%
	\begin{abstract}
In our previous work [Adv. Appl. Math. Mech., 2017, 9: 307-323],
we proposed a novel optimization algorithm, the hill-climbing method with a stick (HiCS), to address the unconstrained optimization. Numerical
results have been demonstrated many satisfactory properties.
However, there exist two unsolved issues: convergent
analysis and application to high-dimensional problems. 
In this paper, we will give a rigorous theory to guarantee the finite-step
convergence property by introducing a new definition of the suspected
extreme point. Meanwhile, an economic sampling
strategy based on the regular simplex is
developed to treat high-dimensional optimization. Finally, 
the efficiency of the improved HiCS method is demonstrated by several
high-dimensional examples.
	\end{abstract}
	%%%%% end %%%%%%%%%%%
	
	%%%%% AMS/PACs/Keywords %%%%%%%%%%%
	%\pac{}
	\ams{90C56, 90C59, 65K05.%The information of the AMS subject classification can be found in http://mathscinet.ams.org/msc/msc2010.html
	}
	\keywords{
Hill-climbing method with a stick, Finite-step convergence, 
Simplex, High-dimensional problems}
	
	%%%% maketitle %%%%%
	\maketitle
	
\section{Introduction}
\label{sec:intro}

\textcolor{blue}{
Classic optimization approaches are developed to find the equilibrium point. }
These optimization approaches can be divided into broadly two
classes, directional search and model-based\,\cite{sun2006optimization,
nocedal2006numerical, conn2009introduction}.  Directional search algorithms first
determine the search direction and then the step length along with the search
direction. Model-based approaches construct and utilize a related simple model to
approximate the original problem in a trust region to guide the search process. 
\textcolor{blue}{
Finding equilibrium points in a proper search domain is important 
during the optimization process.
}
Inspired by the behavior of the blind for climbing hill, we
proposed a new derivative-free approach, the hill-climbing method with a stick
(HiCS), to treat unconstrained optimization problems in
our previous work\,\cite{huang2017hill}.
The main idea of the HiCS, at each search step, is comparing
function values on a surface surrounding the current iterator.
It requires a comparison of function values, and does not need the search direction
or construct a surrogate model in a trust region.
\textcolor{blue}{
The HiCS algorithm is different from the direct search methods in
derivative-free optimization. The direct search methods choose a set of nonzero
vectors deterministically or stochastically as search directions, then try to find a
replaceable point along search directions with a step size until a certain stopping
criterion is met.  
The HiCS algorithm tries to find a neighbourhood that contains equilibrium points
rather than directly finding these equilibrium points.
}
Numerical results have been demonstrated that the HiCS has many 
satisfactory properties, including being easy to implement,
a unique parameter to be modulated, and having the capacity
for finding the local and global maximum. 
However, there are two unsolved problems in the
previous work, including a rigorous theoretical explanation and
the treatment of high-dimensional optimization.
This paper will give the convergence analysis and related
properties of this algorithm by introducing a new concept. 
Meanwhile, a new strategy will be proposed to sample the search
surface to address high dimension optimization problems.

In the following, we will briefly introduce the HiCS algorithm and
prove its finite-step convergence in Sec.\,\ref{sec:algorithm}. 
The algorithm implementation is presented in Sec.\,\ref{sec:implement}.
In particular, the new sampling strategy using the regular simplex is
also given in this section.
The numerical experiments including high dimensional optimization
problems are showcased in Sec.\,\ref{sec:experiment}. 
Finally the conclusion and discussions are given in Sec.\,\ref{sec:conclusion}.

\section{HiCS algorithm and convergence analysis}
\label{sec:algorithm}

Before going further, a short introduction of the
HiCS method is necessary.
We consider an unconstrained optimization problem 
\begin{align}
	\min_{x\in\Omega\subset\mathbb{R}^d} f(x),
	\label{}
\end{align}
where the objective function $f(x):\bbR^d\rightarrow \bbR$.
Let $\rho$ be the search radius, $O(x_k, \rho)=\{x:
\|x-x_k \|=\rho\}$ be the search surface at the
$k$-th iteration with radius $\rho$. $\|\cdot \|$ is the common
norm in $\bbR^d$ space.  $U(x_k, \rho)$ is the neighbourhood of
$x_k$ with radius of $\rho$.  To illustrate the algorithm
more accurately, an useful concept of the \textit{suspected extreme point}
is introduced.

\textcolor{blue}{
\begin{definition}	
	For a given objective function $f(x)$ and a positive constant 
	$\rho>0$, $\tilde{x}$ is a \textit{suspected extreme point} if
	$f(\tilde x)\leq f(x)$ or $f(\tilde x)\geq f(x)$, for each $x \in O(\tilde{x},\rho)$.
	If $f(\tilde x) \leq f(x)$ for all $x\in O(\tilde{x},\rho)$,
	$\tilde{x}$ is the \textit{suspected minimum point (SMP)}.
\end{definition}
\begin{remark}
	Denote $\overline{U}(\tilde{x}, \rho)$ is the closure of $U(\tilde{x}, \rho)$, i.e., 
	$\overline{U}(\tilde{x}, \rho) = U(\tilde{x}, \rho)\cup O(\tilde{x}, \rho) $.
	If $f(x)$ is continuous in domain $\overline{U}(\tilde{x}, \rho)$,
	$\tilde x$ is a SMP, then there exists at least a minimizer in $U(\tilde x,
	\rho)$ or $f(x)$ is constant on $\overline{U}(\tilde x, \rho)$.
\end{remark}
}


%Obviously, $\tilde{x}$ is a SMP if $\tilde{x}$ is a minimizer in
%the neighborhood of $U(\tilde{x}, \rho)$. The opposite is not always true.
With these notations, the HiCS algorithm can be presented
as the Algorithm \ref{alg:HiCS}.
\begin{algorithm}[H]
	\caption{Hill-Climbing method with a stick (HiCS)}
	\label{alg:HiCS}
\begin{algorithmic}[1]
	\STATE \textbf{Initialization:} Choose $x_0$ and $\rho$.
	\STATE \textbf{For} $k=0,1,2,\dots$
	\STATE \hspace{0.5cm} 
	Find $\bar{x}=\argmin\limits_{y\in O(x_k,~ \rho)} f(y)$.
			\\
	\STATE \hspace{0.5cm} If $f(\bar x)<f(x_k)$, then set $x_{k+1}= \bar{x}$.
		  \\
	\STATE \hspace{0.5cm} Otherwise, declare that 
		   a SMP is found, and end the iteration.
\end{algorithmic}
\end{algorithm}
The approximation error of the HiCS algorithm
can be measured by the distance between the SMP and a minimum.
When the HiCS converges, the approximation error is smaller than the search radius
$\rho$.
From our experience, the HiCS approach usually terminates in finite steps. It is a satisfactory property. In what follows, we will
prove the finite-step convergence with some mild conditions.

\begin{theorem}[Finite-step convergence]
	\label{thm:fsc}
	Suppose that objective function $f(x)$ is continuous and the
	search domain $\Omega$ is a compact set.
	If there are not two SMPs $x_*$ and $x^*$ satisfying 
	$\|x_*-x^*\|=\rho$ and $f(x^*)=f(x_*)=\alpha$.
	Then Algorithm \ref{alg:HiCS} converges in finite steps.
\end{theorem}
\begin{proof}
	Assume that the HiCS method generates an infinite pair sequence
	$\{x_n, f(x_n)\}_{x=0}^{\infty}$. From these assumptions,
	it is obvious $f(x)$ is bounded. The decreasing sequence
	$\{f(x_n)\}_{n=0}^\infty$ converges, and the bounded
	$\{x_n\}_{n=0}^\infty$ has a convergent subsequence 
	$\{x_{n_k}\}_{k=0}^\infty$. Assume that $f(x_n)\rightarrow
	\alpha$ and $x_{n_k}\rightarrow x^*$. Obviously $x^*$ is a SMP.
	
	According to the subsequence
	$\{x_{n_k}\}_{k=0}^\infty$, we can always choose an another
	bounded subsequence $\{x_{n_k -1}\}_{k=0}^\infty \subset
	\{x_n\}$ satisfying $\|x_{n_k - 1}-x_{n_k}\|=\rho$. 
	Due to the boundedness of iteration
	sequence, $\{x_{n_k-1}\}_{k=0}^\infty$ has a convergent
	subsequence $\{x_{n_{m}}\}_{m=0}^\infty$. Let $x_{n_m}
	\rightarrow x_*$ when $m\rightarrow \infty$. $x_*$ is also a SMP.
	From the $\{x_{n_m}\}$, we can find a subsequence
	$\{x_{n_{m}+1}\}\subset \{x_{n_k}\}$ which satisfies
	$\|x_{n_m}-x_{n_{m}+1}\|=\rho$, and $x_{n_{m}+1}\rightarrow x^*$
	($m\rightarrow \infty$).
	Obviously, $\|x^*-x_*\|=\rho$, and $f(x^*)=f(x_*)=\alpha$ 
	which clearly contradicts the assumption.
\end{proof}

\section{Algorithm implementation}
\label{sec:implement}

As mentioned above, the HiCS algorithm can converge in  
finite steps with mild assumptions for a given search radius $\rho$.
The search surface $O(x_k,\rho)$ in
each iteration shall be discretized in the numerical implementation.
Without a priori information of the objective function,
the discretization principles for $O(x_k,\rho)$
should include symmetric and uniform distribution and as few discretization points as possible.
Our previous results have demonstrated that the uniformly distributed
discretization method based on the spherical coordinate
can be used to find the SMP\,\cite{huang2017hill}. 
The generated discretization points are as large as $2m^{d-1}$ in
each iteration, $m$ is the number of refinement, $d$ is the dimensions of
optimization problems. It restricts the
application to high-dimensional problems. 
To overcome this limitation, it needs to develop a new
strategy to discretize $O(x_k,\rho)$ with fewer discretization points
but still satisfying these properties.
A reasonable requirement for discretization points should be
linear or quasi-linear growth as the problem's dimension increases.
In this work, we will use the regular simplex and its rotations to
discretize the search surface $O(x_k,\rho)$. 
The computational complexity grows linearly as the
dimension of optimization problems increases.

The $d$-dimension regular simplex is a congruent polytope of
$\mathbb{R}^d$ with a set of points $\{a_1,\dots,a_d,a_{d+1}\}$,
and all pairwise distances $1$.
Its Cartesian coordinates can be obtained from the following two properties:
\begin{enumerate}
	\item For a regular simplex, the distances of its vertices 
		$\{a_1,\dots,a_d,a_{d+1}\}$ to its center are equal.
	\item The angle subtended by any two vertices of the 
		$d$-dimension simplex through its center is
		$\arccos(-1/d)$.
\end{enumerate}
In particular, the above two properties can be implemented
through the Algorithm \ref{alg:simplex}.
\begin{algorithm}[!htpb]
	\caption{Generate $d$-D regular simplex coordinates} 
	\label{alg:simplex}
\begin{algorithmic}
	\STATE Given a $d\times(d+1)$-order zero matrix $x(1:d,1:d+1)$
	\STATE Let $x(:,1) = (1,0,\dots,0)$
	\FOR {$i=2:1:d$}
	\STATE $x(i,i)=\sqrt{1-\sum_{k=1}^{i-1} [x(k, i)]^{2}}$
		\FOR {$j=i+1:1:d+1$}
		\STATE $x(i,j)
		=-\frac{1}{x(i,i)}\Big[\frac{1}{d}+x(1:i-1, i)^T \cdot
		x(1:i-1, j\,)\Big]$
		\ENDFOR
	\ENDFOR
	\STATE Output the column vectors, and let $a_j=x(:,j)$,
	$j=1,2,\dots,d+1$ 
\end{algorithmic}
\end{algorithm}

If the HiCS method does not find a better state at the initial given regular
simplex, more points can be added to discretize $O(x_k, \rho)$ by 
rotating regular simplex.
For a given rotation angle
$\theta=(\theta_1,\theta_2,\dots,\theta_{d})$, the
rotation matrix $\bm Q$ is given as 
\begin{equation}
\begin{aligned}
	{\bm Q} = 
	 \prod_{i=2}^{d-1} &
\bordermatrix{
  &  &       &  & 		   & i &		   &  &  & \cr
  & 1&       &  & 		   & \vdots  & 		   &  &  &  \cr
  &  & \ddots&  & 		   & \vdots  & 		   &  &  &  \cr
  &  &       & 1&          & \vdots  & 		   &  &  &  \cr
  &  &       &  & \cos \theta_i & 0 & -\sin \theta_i &  &  &  \cr
  &  &       &  &   0	 & 1 &     0     &  &  & \cr 
  &  &       &  & \sin \theta_i & 0 &  \cos \theta_i &  &  &  \cr
  &  &       &  &          &   &           & 1 & &  \cr
  &  &       &  &          &   &           &  & \ddots &   \cr
  &  &       &  &          &   &           &  &  & 1 
}
\\
	& \begin{pmatrix}
  \cos \theta_1 & -\sin \theta_1 & 0 &  		&   \\
  \sin \theta_1 & \cos \theta_1  & 0 & 	 	& 	\\
  	0	   &      0    & 1 & 		&   \\
  		   & 		   &   & \ddots &   \\
  		   & 		   &   &   		& 1 
	\end{pmatrix}
	\begin{pmatrix}
  1 &  &  &  		&   \\
    & \ddots  &  & 	 	& 	\\
    &    & 1 & 	0	& 0  \\
    &    & 0 & \cos\theta_d & -\sin\theta_d  \\
    & 	 & 0 &  \sin\theta_d & \cos\theta_d 
	\end{pmatrix}.
\end{aligned}
	\label{}
\end{equation}
Then vertices of new simplex are
\begin{align}
	a_j = \bm{Q}a_j + x_k, \quad j = 1,\dots, d+1.
	\label{}
\end{align}
\begin{figure}[!htbp]
	\centering
	\subfigure[2D regular simplexes]{
%      \includegraphics[scale=0.3]{../figures/2Dsketch.png}
		  \includegraphics[scale=0.08]{../figures/2D1.png}
		  \includegraphics[scale=0.08]{../figures/2D2.png}
		  \includegraphics[scale=0.08]{../figures/2D3.png}
		  \includegraphics[scale=0.08]{../figures/2D4.png}
	  }
	\subfigure[3D regular simplexes]{
	  \includegraphics[scale=0.4]{../figures/3Dsketch.png}
	  }
	\caption{The first regular simplexes of sampling the search set $O(x, \rho)$.}
\label{fig:obset:sketch}
\end{figure}
When without a priori knowledge of objective function, the
uniform distribution of these regular simplexes is another
principle to be obeyed.
%%%% How to rotate
Standard schematic simplexes of 2D and
3D cases are given in Fig.\,\ref{fig:obset:sketch}.
It should be noted that there are also other strategies to
discretize $O(x_k,\rho)$. 
For example, the new adding discretized points 
depend on the known information of objective functions. 

To save computational amount, we choose a dynamic refinement
strategy to discretize the search surface and compare function
values in practice. 
Based on the dynamic refinement strategy, we propose the
practical HiCS, see Algorithm \ref{alg:refined}. The computational amount is not
larger than $m_{\max}(d+1)$ in each iteration, which is linearly
dependent on the dimension of optimization problems,
$m_{\max}$ is the maximum number of rotation.
Whence, it allows us to treat high-dimensional optimization problems.

\begin{algorithm}[H]
	\caption{Practical HiCS}
	\label{alg:refined}
\begin{algorithmic}[1]
	\STATE Input $x_0$, $\rho$, and $m_{\max}$
	\FOR {$k=0,1,2,\cdots$}
		\STATE Set $m=0$
		\IF {$m\leq m_{\max}$}
			\STATE Discretize $O(x_k,\rho)$ to obtain $O^m_h(x_k,\rho)$
			\IF {$\exists x_j \in O^m_h(x_k,\rho)$, s.t.  $f(x_j)<f(x_k)$}
				\STATE Set $x_{k+1}=x_j$, and $m=m_{\max}+1$
			\ELSE
				\STATE Set $m = m+1$
			\ENDIF
		\ELSE
			\STATE Declare that find a SMP, end program
		\ENDIF
	\ENDFOR
\end{algorithmic}
\end{algorithm}

It is evident that once the HiCS converges,
the search space shrinks to a ball with the radius $\rho$, and
more significantly, the convergent ball contains a SMP.
We will demonstrate this by several numerical experiments in
Sec.\,\ref{sec:experiment}.
Note that the convergent result provides a
the good initial value for other optimization approaches, including 
directional search and model-based algorithms.

We can also adjust the search radius $\rho$ in HiCS method 
to improve the approximation precision as done in our previous
work\,\cite{huang2017hill}. Algorithm\,\ref{alg:AHiCS} gives the
process by adaptively changing $\rho$ when Algorithm\,\ref{alg:refined} fails
to find $f(\bar{x})<f(x_k)$, $\bar{x}\in O(x_k, \rho)$ for a fixed $\rho$.
The approximation distance between convergent point and a SMP is
improved when Algorithm\,\ref{alg:AHiCS} converges when $\eta <1$.
Certainly, the search surface can be expanded by setting
control factor $\eta>1$ if required. 
The Algorithm\,\ref{alg:AHiCS} can be restarted by
fixed $k$ iterations or by other criterions
with different search radius.

\begin{algorithm}[H]
	\caption{Adaptive HiCS}
	\label{alg:AHiCS}
\begin{algorithmic}[1]
	\STATE Input $x_0$, $\rho$, $m_{\max}$,
	$\varepsilon$ and $\eta>0$
	\IF {  $\rho>\varepsilon$}
	\FOR {$k=0,1,2,\cdots$}
		\STATE Set $m=0$
		\IF {$m\leq m_{\max}$}
			\STATE Discretize $O(x_k,\rho)$ to obtain $O^m_h(x_k,\rho)$
			\IF {$\exists x_j \in O^m_h(x_k,\rho)$, s.t.  $f(x_j)<f(x_k)$}
			\STATE Set $x_{k+1}=x_j$, 
				and $m=m_{\max}+1$
			\ELSE
				\STATE Set $m = m+1$
			\ENDIF
		\ELSE
			\STATE Set $\rho=\eta\rho$
		\ENDIF
		\STATE Set $k=k+1$
	\ENDFOR
\ENDIF
\end{algorithmic}
\end{algorithm}


\section{Numerical results}
\label{sec:experiment}

In this section, we choose two kinds of high-dimensional optimization functions, including the unimodal Gaussian function, multimodal problems, to demonstrate our proposed algorithm's performance. 
These objective functions are all differentiable. However, it is
emphasized that the HiCS method can be applied to non-differentiable problems. 
In Algorithm\,\ref{alg:refined}, the discretized points of 
search set in each iteration are $m(n+1)$, $n$ is the dimension
of objective function. If not specified, the maximum number of
rotation $m=32$.

\subsection{The unimodal problem: Gaussian function}
\label{subsec:gauss}

The first objective function is the unimodal Gaussian problem
\begin{align}
	f(x) = -20\exp\left(-\sum_{j=1}^d x_j^2 \right),
	\label{eqn:exp1}
\end{align}
which has one minimum $0$ with $f(0)=-20$.
The objective function is differentiable in $\bbR^d$, however,
it quickly diffuses out towards zero out of the upside-down ``bell''. 

We first investigate the convergent property of HiCS
method for $10$ dimensional Gaussian function using 
$30$ experiments.
In the set of experiments, the search radius $\rho$ is fixed as
$0.3$, start points are all randomly generated in the space $[-1,
1]^{10}$.  For each experiment, the HiCS method indeed converges and
captures a neighborhood of the peak
$0$ in finite iterations as Theorem \ref{thm:fsc} predicts.
Fig.\,\ref{fig:exp1:randInit} gives the required iterations for
convergence in the $30$ numerical experiments.
\begin{figure}[!htbp]
	\centering
	  \includegraphics[scale=0.2]{../figures/gauss10Drandr0_3.png}
	  \caption{
	  The required iteration steps of the 
	  HiCS algorithm for the Gaussian function
	  \eqref{eqn:exp1} in $30$ runs with randomly generated start points
	  in the space $[-1, 1]^{10}$, and $\rho=0.3$. 
	  The flat dashed line shows the average.} 
	  \label{fig:exp1:randInit}
\end{figure}
In these $30$ runs, the average iterations of convergence is
$20.5$, while the maximum is $27$, and the minimum is $9$.

Then we decrease the search radius $\rho$ to $0.1$ to observe the
HiCS method's behavior in $30$ numerical tests. 
The initial values are also randomly generated in the same region.  
The required iterations for convergence are given in
Fig.\,\ref{fig:exp1:randInitr0_1}.
In these $30$ runs, the average iterations of convergence are
$77.2$, while the maximum is $121$, and the minimum is $54$.
From these results, the HiCS approach all converges in
finite iterations. Meanwhile, it is obvious that the value of
$\rho$ affects the number of iterations. 
\begin{figure}[!htbp]
	\centering
	  \includegraphics[scale=0.2]{../figures/gauss10Drandr0_1.png}
	  \caption{
	  The required iteration steps of the 
	  HiCS algorithm for the Gaussian function
	  \eqref{eqn:exp1} in $30$ runs with randomly generated start points
	  in the space $[-1, 1]^{10}$, and $\rho=0.1$. 
	  The flat dashed line shows the average.} 
	  \label{fig:exp1:randInitr0_1}
\end{figure}


In the following, we apply the adaptive HiCS algorithm to $1000$ dimensional
Gaussian function. The initial value is randomly generated in
domain $[-1000,1000]^{1000}$, the initial search radius $\rho_0 = 2.0$,
and control factor $\eta=(\sqrt{5}-1)/2$. 
Fig.\,\ref{fig:gauss:1000D} presents the iteration process.
The left image in Fig.\,\ref{fig:gauss:1000D} gives
the difference between $f(x_k)$ and $f(0)=-20$. 
The right one in Fig.\,\ref{fig:gauss:1000D} 
plots the changes of search radius $\rho$ and $\ell^2$-distance between
the iterator and the global minimizer $x^*=0$, where $\|x\|_{\ell^2}=\left(
(\sum_{i=1}^d x_i^2) / d\right)^{1/2}$.
From these results, it can be found that the HiCS is convergent
for each $\rho$. Based on the iteration, the adaptive HiCS method
can approximate the global minimum by decreasing the search radius $\rho$. 
Meanwhile, during the iteration, the global minimizer is always in the
search neighbourhood. 
\begin{figure}[!htbp]
	\begin{minipage}[b]{0.5\linewidth}
	\centering{
	  \includegraphics[scale=0.25]{../figures/gauss1000D.png}
	  }
%    \centerline{(a) }
	\end{minipage}
	\begin{minipage}[b]{0.5\linewidth}
	\centering{
	  \includegraphics[scale=0.25]{../figures/gauss1000D_dist.png}
	  }
%    \centerline{(b) }
	\end{minipage}
	  \caption{The iteration process of the adaptive HiCS method to 1000
	  dimensional Gaussian function. 
	  Start point is randomly generated in the space $[-1000,
	  1000]^{1000}$, $\rho=2.0$ and control factor
	  $\eta=(\sqrt{5}-1)/2$. The left plot is the energy
	  difference, and the right one is the search radius and 
	  the $\ell^2$ distance between the current iterator and the
	  global minimizer $x^*=0$.
	  } 
	  \label{fig:gauss:1000D}
\end{figure}

\subsection{The multimodal problems: Ackley and Arwhead functions}
\label{subsec:minmulit}

The second test objective function is the Ackley
function\,\cite{dieterich2012empirical} which is a widely used 
benchmark function for testing optimization algorithms.
The expression of the Ackley function can be written as
\begin{align}
	f(x) =
	-20\cdot\exp\left(-\frac{1}{5}\cdot\sqrt{\frac{1}{d}\sum_{i=1}^d
	x_i^2}\right)-
	\exp\left(\frac{1}{d}\sum_{i=1}^d \cos(2\pi x_i)\right)+20+e,
	\label{eqn:ackley}
\end{align}
where $n$ is the dimension.
Ackley function has many local minima and a unique global
minimum of $0$ with $f(0)=0$, which poses a risk for
optimization algorithms to be trapped into one of local
minima, such as the traditional hill-climbing method\,\cite{back1996evolutionary}.
Our previous result has shown that the HiCS method can capture
different local minimizer and the global minimizer for 2
dimensional problem through the choice of different $\rho$\,\cite{huang2017hill}.
In this subsection, we will apply the improved HiCS
algorithm to higher dimensional Ackley function. 
In the following simulation, the control factor
$\eta=(\sqrt{5}-1)/2$.

\begin{table}[!hbpt]
\caption{
The successful number $N_s$ of capturing the global minimizer for
each different initial search radius $\rho_0$ when applying the
adaptive HiCS method to $100$ dimensional Ackley function from 100
time numerical experiments. 
The initial values are randomly generated in $[-10,10]^{100}$.
}
\label{tab:ackley100D:AHiCS}
\begin{center}
\begin{tabular}{|c|c|c|c|c|c|c|c|c|c|c|}
 \hline
  $\rho_0$  & 2.0 & 1.8 & 1.6 & 1.4 & 1.2 & 1.0 & 0.8 & 0.6 & 0.4 & 0.2 
 \\\hline
  $N_s$     & 98  & 99  & 97  & 73  & 93  & 100 & 99  & 84  & 76 & 57 
\\\hline \hline
 $\rho_0$ & 0.1 & 0.09 & 0.08 & 0.07 & 0.06 & 0.05 & 0.04 & 0.03& 0.02 & 0.01
 \\\hline
  $N_s$& 75 & 79 & 72 & 69 & 84 &86 & 52 & 0 & 0 & 0
\\ \hline
\end{tabular}
\end{center}
\end{table}
We first take $100$ dimensional Ackley function as an example to
test the performance of our proposed algorithm for finding minimizers. 
We run adaptive HiCS method 100 times for each different initial
search radius $\rho_0$ from $0.01$ to $2.0$.
The start points are all randomly generated in $[-10,10]^{100}$.
The convergent criterion is the search radius smaller than $10^{-10}$.
Tab.\,\ref{tab:ackley100D:AHiCS} gives the successful number
$N_s$ of capturing the global minimizer.  
When the algorithm is successful, the distance between the
convergent iterator and the global minimizer is smaller than the
search radius $\rho < 10^{-10}$.
From these results, it is easy to find that our method can
approximate the global minimizer. 
The value of $\rho_0$ heavily affects the probability of
obtaining the global minimizer. 
When $\rho_0 > 0.04$, the adaptive HiCS can find the global
minimizer with high probability. 
When $\rho_0$ is about $0.04$, 
the successful probability is falling quickly to about $50\%$. 
As $\rho_0$ decreases to smaller than $0.03$, the HiCS could not
find the global minimizer.
Besides, it should be pointed out that these so-called unsuccessful
experiments have obtained other local minimizers. 



We continue to apply the adaptive HiCS method to $2500$ dimensional
Ackley function. The initial search radius is $\rho_0=3.5$, and
the initial position is generated randomly in $[-10,10]^{2500}$. 
The iteration process is presented in Fig.\,\ref{fig:ackley2500D:AHiCS}. 
For such a high dimensional optimization problem, the iteration
behavior is similar to previous numerical experiments. 
When $\rho=3.5$, the HiCS method costs $90$ steps to achieve convergence.
By further shrinking search radius, the adaptive HiCS can
capture global minimizer. As one can see from 
Fig\,\ref{fig:ackley2500D:AHiCS}, the global minimizer always
locates in the search neighbourhood in this case. 
It demonstrates that the HiCS has the capacity of hupping the
local basin even for such a high dimensional problem.
\begin{figure}[!htbp] 
	\centering
	\includegraphics[scale=0.25]{../figures/ackley2500D.png}
	\includegraphics[scale=0.25]{../figures/ackley2500D_dist.png}
	  \caption{The iteration process of the adaptive HiCS method to 2500
	  dimensional Ackley function with initial search
	  radius $\rho_0=3.5$. Start point is randomly generated in the
	  space $[-10, 10]^{2500}$. } 
	\label{fig:ackley2500D:AHiCS}
\end{figure}

The last benchmark example is the Arwhead function, which has been
also used by Powell to test the NEWUOA derivative-free
method\,\cite{powell2006newuoa}. The expression of the Arwhead function is
\begin{align}
	f(x) = \sum_{i=1}^{d-1}[(x_i^2+x_n^2)^2 - 4 x_i +3].
	\label{}
\end{align}
The least value of $f$ is zero, which occurs when the minimizer
$x^*$ take the values $x_j=1$, $j=1,2,\dots,d-1$ and $x_d=0$. 
We directly apply the adaptive HiCS method ($\eta=0.5$) to $1000$
dimensional Arwhead function.
The starting vector is given by $x_j^{(0)}=1$, $j=1,2,\dots,d$, as
Powell done in Ref.\,\cite{powell2006newuoa}
The initial search radius $\rho_0=3$ and $\eta=(\sqrt{5}-1)/2$.
\begin{figure}[!htbp]
	\centering
	  \includegraphics[scale=0.25]{../figures/arwhead1000D.png}
	  \includegraphics[scale=0.25]{../figures/arwhead1000D_dist.png}
  \caption{The iteration process of the adaptive HiCS method
  ($\eta=(\sqrt{5}-1)/2$) to the $1000$ dimensional Arwhead function.}
	\label{fig:arwhead}
\end{figure}

Fig.\,\ref{fig:arwhead} gives the iteration process of
applying the adaptive HiCS algorithm to $1000$ dimensional Arwhead function.
The sequences of function values and iterators approximate the global minimum and the global minimizer. 
The function value always decreases as the proposed algorithm
indicates. While the distance $\|x^{(k)}-x^*\|_{\ell^2}$
demonstrates more interesting phenomena. In the beginning, the
search radius $\rho$ is larger than the distance, which means the global
minimizer $x^*$ is in the search neighborhood. Then when the distance is
about $1.67\times 10^{-1}$, the $\rho$ is smaller than the
distance, which indicates $x^*$ is not in the search neighborhood. 
It means that the iterator locates in the valley of a local minimizer. 
However, as iteration evolves, the HiCS algorithm can jump out of
the local energy trap well and again contains the global
minimizer in the search region.



\section{Conclusion}
\label{sec:conclusion}

Inspired by the hill-climbing behavior of the blind, we have
proposed a new derivative-free method to unconstrained
optimization problems in our previous work\,\cite{huang2017hill}. 
This paper establishes a rigorous mathematical theory of the HiCS
algorithm, which theoretically guarantees the finite-step convergence
under mild conditions. Numerical results also have demonstrated
the satisfactory property. 
In practice, the computational complexity of the HiCS algorithm mainly
depends on the discretized strategy on search boundaries.
We proposed a new simplex
discretization method to save computational amount to address high-dimensional problems in this work. Using the
simplex method, the number of function evaluations is linearly
dependent on the dimension of problems,
which allows us to solve high-dimensional optimization problems. 
Finally, we demonstrate the efficiency of our proposed algorithm
by solving several higher-dimensional benchmark problems.


\section*{Acknowledgments}
This work is supported by the National Natural Science Foundation of China (11971410,
11771368) and Project for Hunan National Applied Mathematics Center of Hunan
Provincial Science and Technology Department (2020ZYT003).
KJ is partially supported by the Key Project of the Education Department of 
Hunan Province of China (19A500). 


\begin{thebibliography}{99}

\bibitem{sun2006optimization}
W.~Y.~Sun and Y.~Yuan,
Optimization theory and methods: nonlinear programming,
New York: Springer, 2006.

%\bibitem{conn2000trust}
%A.~R.~Conn, N.~I.~M.~Gould and P.~L.~Toint,
%Trust region methods, Philadelphia: SIAM, 2000.

\bibitem{nocedal2006numerical}
J.~Nocedal and S.~J.~Wright,
Numerical optimization, 
Berlin: Springer-Verlag, 2nd ed., 2006.

\bibitem{conn2009introduction}
A.~R.~Conn, K.~Scheinberg and L.~N.~Vicente,
Introduction to derivative-free optimization,
Philadelphia: SIAM, 2009.

\bibitem{huang2017hill}
{Y.~Huang and K.~Jiang,}
{Hill-climbing algorithm with a stick for unconstrained optimization problems},
{Adv. Appl. Math. Mech.},
2017, 9: 307--323.

\bibitem{dieterich2012empirical}
J.~M.~Dieterich and B.~Hartke, 
Empirical review of standard benchmark functions using evolutionary global optimization,
{Appl. Math.} 2012, 3: 1552--1564.

\bibitem{back1996evolutionary}
T.~B{\"a}ck, 
Evolutionary algorithms in theory and practice: evolution
  strategies, evolutionary programming, genetic algorithms,
Oxford University Press, 1996.

\bibitem{powell2006newuoa}
M.~J.~D.~Powell,
The NEWUOA software for unconstrained optimization without derivative, in
Large-Scale Nonlinear Optimization , eds. G.~Di~Pillo
and M.~Roma, Springer (New York), 2006, 255--297.


%%\bibitem{rios2013derivative}
%%L.~M.~Rios and N.~V.~Sahinidis,
%%Derivative-free optimization: a review of algorithms and comparison
%%  of software implementations.
%%{J. Global Optim.}, 2013, 56: 1247--1293.
%%
%\bibitem{powell2000uobyqa}
%M.~J.~D.~Powell, UOBYQA: unconstrained optimization by quadratic
%approximation, Technical Report DAMTP NA2000/14, CMS, University
%of Cambridge, 2000.
%
%\bibitem{powell2002trust}
%M.~J.~D.~Powell, On trust region methods for unconstrained
%minimization without derivatives, Technical Report DAMTP
%NA2002/NA02, CMS, University of Cambridge, February 2002.
%
%%\bibitem{wu2009heuristic}
%%T.~Wu, Y.~Yang, L.~Sun, and H.~Shao, A heuristic
%%iterated-subspace minimization method with pattern search for
%%unconstrained optimization, 
%%Comput. Math. Appl., 2009, 58: 2051-2059.
%%
%\bibitem{zhang2014sobolev}
%Z.~Zhang, Sobolev seminorm of quadratic functions with
%applications to derivative-free optimization, Math. Program.,
%2014, 146: 77-96.
%
%\bibitem{michalewicz2004how}
%Z. Michalewicz and D. B. Fogel, How to solve it: modern
%heuristics, Springer, 2004.
%
%%\bibitem{lecun2015deep}
%%Y.~LeCun, Bengio, Y.~Hinton, G.~Hinton, Deep learning, Nature,
%%2015, 521: 521-436.
%%
%%\bibitem{hooke1961direct}
%%R.~Hooke and T.~A.~Jeeves,
%%``Direct search'' solution of numerical and statistical problems,
%%{J. ACM}, 1961, 8: 212--229.
%%
%\bibitem{lewis2000direct}
%R.~M.~Lewis, V.~Torczon and M.~W.~Trosset,
%Direct search methods: then and now,
%{J. Comput. Appl. Math.},
%2000, 124: 191--207.
%
%\bibitem{nelder1965simplex}
%J.~A.~Nelder and R.~Mead,
%A simplex method for function minimization,
%{Comput. J.}, 1965, 7: 308--313.
%
%\bibitem{torczon1997convergence}
%V.~Torczon,
%On the convergence of pattern search algorithms,
%{SIAM J. Optim.}, 1997, 7: 1--25.
%
%\bibitem{kolda2003optimization}
%T.~G.~Kolda, R.~W.~Lewis and V.~Torczon,
%Optimization by direct search: new perspectives on some classical
%and modern methods,
%{SIAM Rev.}, 2003, 45: 385--482.
%
%\bibitem{dennis1991direct}
%J.~E.~Jr Dennis and V.~Torczon,
%Direct search methods on parallel machines,
%{SIAM J. Optim.}, 1991, 1: 448--474.
%
%\bibitem{gratton2015direct} 
%S.~Gratton, C.~W.~Royer, L.~N.~Vicente, Z.~Zhang, Direct search
%based on probabilistic descent, SIAM J.~Optim., 
%2015, 25:1515-1541.
%
%\bibitem{russell2010artificial} 
%S.~J.~Russell and P.~Norvig,  Artificial intelligence: a modern
%approach, 3rd ed., Prentice Hall, 2010.
%
%\bibitem{dennis1987optimization}
%J.~E.~Jr Dennis and D.~J.~Woods,
%Optimization on microcomputers: The Nelder-Mead simplex algorithm.
%In: New computing environments: microcomputers in large-scale
%computing, A.~Wouk ed., Philadelphia: SIAM, 1987.
%
%%\bibitem{lukvsan2010modified}
%%L.~Luk{\v{s}}an, C.~Matonoha, J.~Vlcek,
%%Modified CUTE problems for sparse unconstrained optimization,
%%Techical Report,
%%2010, 1081.
%
%%\bibitem{algopy}
%%https://pythonhosted.org/algopy/index.html
%
%%\bibitem{andrei2008unconstrained}
%%N.~Andrei, 
%%An unconstrained optimization test functions collection,
%%Adv. Model. Optim.,
%%2008, 10: 147--161.


\end{thebibliography}
	
\end{document}
